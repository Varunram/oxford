\documentclass{article}
\usepackage[utf8]{inputenc}
\usepackage{geometry}
\usepackage[final]{graphicx}
\usepackage{grffile}
 \geometry{
 top=30mm,
 left=1.5in,
 right=1.5in
 }
\PassOptionsToPackage{hyphens}{url}\usepackage{hyperref}


\title{Kratos: blockchain solution for data privacy, literacy, and student agency in a data-driven educational ecosystem}
\author{Dr.\ Velislava Hillman \quad Varunram Ganesh \quad Geoffrey Martin}
\date{May 2019}

\begin{document}

\maketitle

\textit{Abstract}
Growing digitization has made data ownership an important focus point for institutions and students. Broadly, there are three issues, which require urgent attention for optimization of data privacy, literacy, and utilization. First, schools globally are equivocal about data generated by and about students as a result of the digitization of instruction, learning, and assessment. They lack necessary frameworks for data literacy, data interoperability, and optimization while maintaining privacy and control. Second, the scale, source, and nature of school data makes its interoperability impractical, resulting in an inability to assess the true impact of educational technologies on instruction and learning. Third, while data helps teachers improve pedagogical practices, an increasingly data-driven decision-making process suggests that student dimensions of learning and equitable participation in curriculum design becomes secondary. Finding a balance between increasingly data-driven decision making and student voice and choice is critical for an efficient educational ecosystem. In this paper, we introduce Kratos: an immutable decentralised data management system that provides data privacy and applied data literacy while empowering students with a user interface for data governance and active participation in the educational ecosystem. Using the advantages afforded by blockchain technologies, Kratos provides easy authentication, confidentiality, and accountability for data access, use, and sharing. The objective of Kratos is thus to equip schools and students with the ability to access, manage, control, and understand data and how, why, and by whom data is accessed and used at any one point in time without compromising student agency and privacy. This paper describes proof of concept for a decentralised data management system and its benefits to the educational ecosystem.
\bigbreak

\section{Introduction}

To improve work, school practitioners at local, district, state, and federal levels need data interoperability and educational information from data mining [Gates Foundation, 2015; US Dept of Education, 2012; ACT Policy Report, 2015]. This need originally emanates from the United States federal law which mandates schools at district, state, and federal level to collect student data to serve as 'accountability' metric for assessing school effectiveness [Elementary and Secondary Education Act, 2001; 2010].
\bigbreak
Educational data is important not only for evaluating school quality. It can also contribute to research in learning and advance theory [Baker, 2016]. It can help guide intervention and lead to higher school attainment [Arnold and Pistill, 2012], improved pedagogy [Baker, 2014], and better parental involvement in school matters [Hawn, 2015]. A common goal of educational data is to equip educators with key information that they can act upon and use to the benefit of learners [Baker, 2014]. Educational data can enable an applied data literacy. Furthermore, access to school records provides an opportunity for student agency by enabling learning and participation.
\bigbreak
On the other hand, concerns arise about the pervasiveness of fine-grained data collected about students [Zeide, 2017] and the risk of creating a 'permanent record' that can impede upon learners' futures [Cody, 2013]. The use of online platforms, programs, and applications in schools often provided by for-profit vendors, many of whom have unclear policies about data privacy and third-party sharing [Common Sense Education, 2018], generates continuous stream of data about students' behavior and performance at an unprecedented scale [Zeide, 2017]. Big data complicates traditional understanding of what constitutes sensitive information and what information serves an educational purpose. Data continues to drive decision-making of practitioners, vendors, and policy makers, marginalizing student dimensions of learning and equitable participation in the curriculum design.
\bigbreak
Interoperability challenges between vendors and schools at local, district, state, or federal level further pose barriers not only to cohesive data management and sharing but also lacks technological privacy infrastructure and accountability [Zeide, 2014; Common Sense Media Research, 2018]. While digitizing and collecting student data is not new [Fitzgerald, 2014], technological developments like cloud computing and the growing use of web applications, learning management systems that gather fine-grained student data, amplify the concerns about data transfer, storage, use, and analysis [Sultan, 2010].
\bigbreak
Increasingly digitized learning environments enable constant data collection and algorithm-based assessment which change core school functions of teaching, assessment, and accreditation [Zeide, 2017]. Digital systems for teaching and assessment drive data-based decision-making and enable micromanagement of students that can further pose restrictions on teacher autonomy and student (and parent) participation and willingness to challenge education decision-making. Interactive learning environments allow students to manage coursework online, take tests, and use a variety of online learning material. Schools deploy management systems for sorting and storing reporting and operational data. While some data is stored on local servers at school district level, other school data is managed by third-party vendors [Zhang, Li and Hao, 2015], who may consider data proprietary and thus inaccessible. This puts additional barriers to effective data sharing and comprehensive understanding of school data.
\bigbreak
The present challenges of fragmented data access and use, lack of student agency, accountability, and auditability to data access and use demand a socio-technological solution, one which can ensure that educational data can be used to help improve school processes while student privacy, agency, and future opportunities are not diminished.
\bigbreak
In this work we propose the development of a structure for student data management, privacy, accountability, and auditability. We build on top of existing data standards and construct a common data schema for disparate data across different systems. Our system organises these references onto an integral structure of student data and complements the existing educational data standards and interoperability in three distinctive ways. Through network permissioning and proofs of ownership on a distributed ledger, we enable data auditability and accountability. We design data analytic models to integrate with existing school systems and data standards. Lastly, we build a simple user interface that gives students, parents and schools access to disparate data, enabling control over what portion of the data can be shared. Additionally, we propose an application with social functionality for student feedback.
\bigbreak
We present Kratos not as an independent solution to lack of data interoperability, accountability, and privacy but rather as a system that complements existing efforts and solutions developed by the various stakeholders in the educational ecosystem (Data Quality Standards, 2018; Project Unicorn; Bill and Melinda Gates Foundation, 2015).

\section{Existing problems}
To contextualise the complexity related to school data - the opportunities and challenges schools face as a result of the growing digitisation of operation and academic processes - Kratos has partnered with  Cambridge Public School (CPS) district that administer public elementary and high schools in Cambridge, Massachusetts, the Access for Learning (SIF), and the Student Data Privacy Consortium. With the help of CPS's Information and Communication Technology Services Chief Information Officer and Database Administrator, we explored data interoperability issues with them along with feedback on potential mitigation schemes:

\begin{itemize}
  \item Data access - CPS is in agreement with over 100 vendors providing education technologies ([7]). The majority of these would mostly provide no direct and comprehensive access to data generated about and by students who make use of their products or services. Where available, data access is  provided in the form of reports - a summary of information which the district database administrator can request and download. Some providers are even less flexible and only offer to give reports once in 45 days making it impossible for teachers to act on feedback that they might infer from the data. On other occasions vendors supply data directly to teachers as well as the students in the form of digital dashboards. Where teachers obtain data it requires further work to convert it into meaningful information upon which the teacher can adjust and plan instruction (Data Quality Campaign, 2018). 
  \item Unified data standardization challenges - The different vendors that CPS partners have their own implementation, most of them not regulated by any standards. The schools themselves follow a common standard defined by the SIF but there is no way to enforce this upon vendors. Even if the vendor agreed to follow a standard defined by the SIF, it is not possible to concretely verify this claim.
  \item Lack of transparency - The vendors do not provide a comprehensive list of the data they collect from students. This makes data audits extremely difficult since we don't exhaustively know what to audit.
  \item Security - It is impossible to verify if vendors are following best practices in data storage and retrieval. This is extremely dangerous since a malicious entity can arbitrarily change or retrieve student records from a poorly maintained database
\end{itemize}

We can thus see that schools are finding it increasingly difficult to obtain data about their own students and are becoming isolated due to a lack of knowledge on how vendors are processing this data.

\section{The need for a solution}

\subsection{Student agency and participation}
The UNESCO framework for educational planning states that "the concern of planners is twofold: to reach a better understanding of the validity of education in its own empirically observed specific dimensions and to help in defining appropriate strategies for change" (Haddad, 1995, pp. 5-6).

While summative and cumulative assessments provide "empirically observed specific dimensions" about student academic performance, student agency and active participation is equally required in order for policy and education to design "strategies for change". Growing use of education technologies enables data-driven decision-making [Gibson et al., 2015]. Technology-mediated instruction and assessment tools with learning analytics functionality track and diagnose student progress. Most educational technologies can interpret and equip educators with information via digital dashboards and 'skill meters'[Baker, 2016; New, 2016]. Fine-grained and continuously accumulated data about student behavior and performance surpasses traditional notions of assessment [Zeide, 2017] posing limitations over student dimensions of learning and equitable participation in the curriculum design.

Our prototype provides a graphical interface for student involvement in and accessibility to school data. While students and equally their parents can become acquainted with any changes and meanings of school data, students can also participate with personal feedback to the learning process. Student participation with personal perspectives and reflection are integral to the learning process. Such reflection is invariably carried out through questionnaire surveys examining school climate [Holahan and Batey, 2019]. Both school climate surveys and the proposed student feedback application provide flexible selection of questionnaires and measurements with a common goal to improve school climate. However, the proposed application enables not only feedback from older students [Ibid.] but from all students, provided that the feedback addresses the common goal. The application further enables student agency and control over the frequency of feedback.

\subsection{Data literacy}
Data literacy is the ability to understand, create, and communicate data as information. Beyond that, in an increasingly data-driven socio-economic systems it becomes imperative not only to understand data but interact with it and participate in the decision-making processes it increasingly exerts its influence. While data provides insight about how systems operate, the inferences drawn from it is without context. Fine-grained data collected over long periods of time can lead to negative impact on individual well-being and pose limitations over future opportunities [Gasser et al., 2018]. The growing digitization of schools create learning environments that enable constant data generation, the access to and use of which becomes hard to control, audit, and account for. The risk of digitized school environments constantly generating data and algorithmic assessments becoming permanent record that at any one point in time may limit individual opportunities leads us to prioritise on developing a socio-technological solution that enables applied data literacy for students at a young age. \textit{a note on Parents for Privacy Consortium and Data Quality Standards who make such efforts to educate students and parents about school data and that we (Kratos) complement these efforts.}

\section{Kratos}
Kratos aims to define a set of principles and guidelines that schools should follow and provides a platform for both schools and students to see what kind of data is being collected by vendors along with ways to tweak data access. The goal for students would be to see how their data is being used and above the age of 18, gain access to this data. We define this as Data Ownership and Visibility- how students can own the data after the age of 18 and how students can see what their data is being used for. The goal for the school would be to make disparate data from different vendors compatible with each other and we define this problem as Data Interoperability.

\subsection{Data ownership}
In conventional systems, data ownership can be proved with the help of an access token but this provides no guarantee on when the owner came into possession of the data. In order to attest ownership at a specific point in time, we need time-stamping services like those described in \hyperref[sec:1]{[1]}, \hyperref[sec:2]{[2]} and \hyperref[sec:3]{[3]}. A time-stamping service requires something to be committed along with the timestamp and Kratos envisions this to be a cryptographic hash which also acts as an access token which vendors can use to access the data.
\bigbreak
The aim of using a cryptographic hash like SHA-3 \hyperref[sec:6]{[6]} is to ensure a uniquely random reference to the data to avoid out of channel data leaks. Kratos suggests using an element of randomness like a salt to generate the hash in order to have the ability to revoke tokens by regenerating randomness. Kratos enforces that all schools encrypt their data before creating access tokens to mitigate the risk of loss/theft of data. Past studies like \hyperref[sec:4]{[4]} and \hyperref[sec:5]{[5]} show that firms are willing to circumnavigate laws to collect data and encrypting data by design ensures that no third party can have access without being granted so explicitly.
\bigbreak
Kratos does not specify at which level data needs to be encrypted - it can be at the school level or at the student level depending on existing frameworks and rules surrounding the school. If the school chooses to encrypt student data on behalf of the student, Kratos enforces that the school use a unique key for each individual to minimize the risk of key theft. In addition to this, Kratos suggests that schools store their encrypted data in a distributed file storage system like IPFS to ensure data redundancy and availability in case of a mishap. Storing data on IPFS also makes it easier to create timestamps since it is sufficient to reference the IPFS pointer instead of potentially hashing the whole data.
\bigbreak
In the event a user wants to revoke access to a particular vendor, he could do so by changing the encryption key or by changing the randomness used to generate the access token. We suggest users don't regenerate encryption keys but Kratos does not enforce this and will provide end users with the option to choose between the two.
\bigbreak
Since schools have their own sets of policies, Kratos doesn't strictly enforce a set of practices for users to follow. This ensures that adoption of Kratos isn't constrained by a certain set of rules. At the same time, Kratos defines a set of minimum requirements to be on board to ensure good practices on date protection are followed.
\bigbreak
Kratos also doesn't specify how and where the commitments need to be stored and leaves it to schools to provide their feedback. Potential solutions include a centralized time-stamping server, a permissioned blockchain with the different schools as the nodes and simple time-stamping commitments to an existing blockchain. All three options have their benefits and constraints and Kratos would arrive at a final recommendation depending on the option the majority of schools are convenient with
\bigbreak

\subsection{Data Interoperability}
A traditional approach to interoperability would be to force a given set of standards on third party vendors, so that they reference them. However, work done by standardization bodies like SIF and CEDS has shown that this is at best partially effective since there is no financial or social incentive for vendors to migrate to the proposed set of standards. Kratos attempts to solve this issue by proposing a solution where different fields described by different vendors can be mapped to a single underlying scheme defined by Kratos and third party vendors would not have to worry about using disparate fields.

\bigbreak
As an example, let us assume the underlying fields defined by Kratos are Grade\_A, Grade\_B and Grade\_C. If a specific vendor has a different name for the same field (example GradeA, GradeB and GradeC), Kratos would ingest the data from the vendor defined fields and convert it to the ones defined by Kratos. This should be done across different formats since the reports database administrators receive right now are formatted as JSON, CSV or are in the form of Excel sheets.

\bigbreak
Kratos proposes that this be done in the form of a templated script that can be written for every vendor. Building a common script for all the vendors poses a challenge since the number of vendors and the different fields and standards they follow are ever changing. This templated script would be triggered automatically by Kratos each time it receives an incoming report and the report itself will be parsed to understand which vendor it originates from.

\bigbreak
The fields that Kratos defines will be adapted from the existing SIF standard followed by schools in Cambridge and would also have routine input from various experts on the subject. The number of fields however would change since SIF has over 700 different fields and it is not possible to accurately map all these fields. Kratos defines the idea of a "bucket" - a collection of data fields collated together as a single entity to make it easier for students and administrators to monitor them. Common examples of buckets would be PII (Personally Identifiable Information), Grade Reports (containing grade reports for each subject) and Attendance Records (recording the class wise attendance of the student). More such buckets will be added after receiving input from students and teachers on what the best ways to classify this data are.

\bigbreak
The models and scripts developed as part of enabling interoperability will be open source and subject to continuous updates. We would also be having a complete code audit before relasing the system in a production environment to ensure the model performs and behaves the way it was designed and intended to.

\bigbreak
The buckets defined by Kratos can be used in multiple ways: students can have a better understanding of how their data is being used, parents can know hwat kind of data is being collected by their children and administrators can ensure that sensitive information is not being shared with third party vendors. After the age of 18, this control would be given to the hands of students and they can decide if they wish to continue sharing data with these parties.

\subsection{User interface}
Visuals, wireframes, mockups, description of these

\section{Discussion and future work}
In this paper we propose Kratos, a data management system, which enables data interoperability for otherwise disparate data, applied data literacy, student agency over their data and participation in designing their education.

- a few paragraphs on data and learning analytics..

- paragraph on Kratos as an add-on, rather than replacement of existing efforts:

\section{References}
\begin{enumerate}
    \label{sec:1}
    \item Todd, P. OpenTimestamps: Scalable, Trustless, Distributed Timestamping with Bitcoin (2016). URL \url{https://petertodd.org/2016/opentimestamps-announcement}
    \label{sec:2}
    \item Szalachowski, P. (2018). Towards more reliable Bitcoin timestamps. arXiv preprint arXiv:1803.09028.
    \label{sec:3}
    \item Massias, H., Avila, X. S., \& Quisquater, J. J. (1999). Design of a secure timestamping service with minimal trust requirement. In the 20th Symposium on Information Theory in the Benelux.
    \label{sec:4}
    \item Zetter, K. Google Collected Data on Schoolchildren without permission (2016). URL \url{https://www.wired.com/2015/12/google-collected-data-on-schoolchildren-without-permission/}
    \label{sec:5}
    \item Dissent, Back-To-School Revolt in Springfield? Employees balk over using Google Drive as evidence of massive privacy breach mounts (2018).  URL \url{https://www.pogowasright.org/back-to-school-revolt-in-springfield-employees-balk-over-using-google-drive-as-evidence-of-massive-privacy-breach-mounts/}
    \label{sec:6}
    \item Bertoni, G., Daemen, J., Peeters, M., Van Assche, G., \& Van Keer, R. (2012). Keccak implementation overview. URL  \url{http://keccak.neokeon.org/Keccak-implementation-3.2.pdf}.
    \label{sec:7}
    \item CPSD. District Agreements Listing (2019). URL  \url{https://sdpc.a4l.org/district_listing.php?districtID=457}
    \label{sec:8}
    \item Colins, A., and Halverson, R. (2018).\textit{ Re-thinking education in the age of technology: The digital revolution and schooling in America}. Teach Zeide, E. (2016).
    \label{sec:9}
    \item 19 Times Data Analysis Empowered Students and Schools: Which Students Succeed and Why?.
    \label{sec:10}
    \item Colins, A., and Halverson, R. (2018).\textit{ Re-thinking education in the age of technology: The digital revolution and schooling in America}. Teachers College Press: New York.
    \label{sec:11}
    \item Cody, A. (2013). Will the data warehouse become every student and teacher's 'permanent record'? \textit{Education Week}, May, 20.
    \label{sec:12}
    \item Gibson, D. C., Webb, M., and Ifenthaler, D. (2015). Challenges of big data in educational assessment.\textit{Proceedings of the IADIS International Conference of Exploratory Learning in Digital Age}, 92-100.
    \label{sec:13}
    \item Sultan, N. (2010). Cloud computing for education: a new dawn? 30 International Journal of Information Management  109.
    \label{sec:14}
    \item Fitzgerald, B. (2014). Data collection isn't new. And it predates common core.Funny Monkey.
    \label{sec:15}
    \item Zeide, E. (2016). 19 Times Data Analysis Empowered Students and Schools: Which Students Succeed and Why?
    
    Project Unicorn. 
    
\end{enumerate}
\begin{figure}
    \centering
    \includegraphics[width=15cm]{{"images/Kratos.jpg}}
    \caption{Kratos Architecture}
    \label{fig:my_label}
\end{figure}
\end{document}
    
