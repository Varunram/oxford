\documentclass{article}
\usepackage[utf8]{inputenc}
\usepackage{geometry}
\usepackage[final]{graphicx}
\usepackage{grffile}
 \geometry{
 top=30mm,
 left=1.5in,
 right=1.5in
 }
\PassOptionsToPackage{hyphens}{url}\usepackage{hyperref}


\title{Kratos: blockchain solution for data privacy, literacy, and student agency in a data-driven educational ecosystem}
\author{Dr.\ Velislava Hillman \quad Varunram Ganesh \quad Geoffrey Martin}
\date{May 2019}

\begin{document}

\maketitle

\textit{Abstract}
Growing digitization has made data ownership an important focus point for institutions and students. Broadly, there are three issues, which require urgent attention for optimization of data privacy, literacy, and utilization. First, schools globally are equivocal about data generated by and about students as a result of the digitization of instruction, learning, and assessment. They lack necessary frameworks for data literacy, data interoperability, and optimization while maintaining privacy and control. Second, the scale, source, and nature of school data makes its interoperability impractical, resulting in an inability to assess the true impact of educational technologies on instruction and learning. Third, while data helps teachers improve pedagogical practices, an increasingly data-driven decision-making process suggests that student dimensions of learning and equitable participation in curriculum design becomes secondary. Finding a balance between data-driven decision making and student voice is critical for an efficient educational ecosystem. In this paper, we introduce Kratos: an immutable decentralised data management system that provides data privacy and applied data literacy while empowering students with a user interface for data governance and active participation in the educational ecosystem. Using the advantages of blockchain technologies, Kratos enables easy authentication and access to data. The objective of Kratos is thus to equip schools and students with the ability to access, manage and control their data and to understand how, why, and by whom data is accessed without compromising student agency and privacy. This paper describes proof of concept for Kratos and its benefits to the educational ecosystem.
\bigbreak

\section{Introduction}

To improve work, school practitioners at local, district, state, and federal levels need data interoperability and educational information from data mining [Gates Foundation, 2015; US Dept of Education, 2012; ACT Policy Report, 2015]. This need originally emanates from the United States federal law which mandates schools at district, state, and federal level to collect student data to serve as 'accountability' metric for assessing school effectiveness [Elementary and Secondary Education Act, 2001; 2010].
\bigbreak
Educational data is important not only for evaluating school quality. It can also contribute to research in learning and advance theory [Baker, 2016]. It can help guide intervention and lead to higher school attainment [Arnold and Pistill, 2012], improved pedagogy [Baker, 2014], and better parental involvement in school matters [Hawn, 2015]. A common goal of educational data is to equip educators with key information that they can act upon and use to the benefit of learners [Baker, 2014]. Educational data can enable an applied data literacy. Furthermore, access to school records provides an opportunity for student agency by enabling learning and participation.
\bigbreak
Increasingly digitized learning environments enable constant data collection and algorithm-based assessment which change core school functions of teaching, assessment, and accreditation [Zeide, 2017]. Digital systems for teaching and assessment drive data-based decision-making and enable micromanagement of students that can further pose restrictions on teacher autonomy and student (and parent) participation and willingness to challenge education decision-making. Schools deploy management systems for sorting and storing data and while some data is stored by schools at the district level, other data is managed by third-party vendors [Zhang, Li and Hao, 2015] who may consider this data proprietary and thus inaccessible. This puts barriers to effective data sharing and comprehensive understanding of school data.
\bigbreak
In this work we propose the development of a structure for student data management, privacy, accountability, and auditability. We build on top of existing data standards and construct a common data schema for disparate data across different systems. Our system organises these references onto an integral structure of student data and complements the existing educational data standards and interoperability in three distinctive ways. Through network permissioning and proofs of ownership on a distributed ledger, we enable data auditability and accountability. We design data analytic models to integrate with existing school systems and data standards. Lastly, we build a simple user interface that gives students, parents and schools access to disparate data, enabling control over what portion of the data can be shared. Additionally, we propose an application with social functionality for student feedback.
\bigbreak
We present Kratos not as an independent solution to lack of data interoperability, accountability, and privacy but rather as a system that complements existing efforts and solutions developed by the various stakeholders in the educational ecosystem (Data Quality Standards, 2018; Project Unicorn; Bill and Melinda Gates Foundation, 2015).

\section{Existing problems}
To contextualise the complexity related to school data - the opportunities and challenges schools face as a result of the growing digitisation of operation and academic processes - Kratos has partnered with  Cambridge Public School (CPS) district that administer public elementary and high schools in Cambridge, Massachusetts, the Access for Learning (SIF), and the Student Data Privacy Consortium. With the help of CPS's Information and Communication Technology Services Chief Information Officer and Database Administrator, we explored data interoperability issues with them along with feedback on potential mitigation schemes:

\begin{itemize}
  \item Data access - CPS is in agreement with over 100 vendors providing education technologies ([7]). The majority of these would mostly provide no direct and comprehensive access to data generated about and by students who make use of their products or services. Where available, data access is  provided in the form of reports - a summary of information which the district database administrator can request and download. Some providers are even less flexible and only offer to give reports once in 45 days making it impossible for teachers to act on feedback that they might infer from the data. On other occasions vendors supply data directly to teachers as well as the students in the form of digital dashboards. Where teachers obtain data it requires further work to convert it into meaningful information upon which the teacher can adjust and plan instruction (Data Quality Campaign, 2018).
  
  \item Lack of data standard compliance - School data frameworks vary across districts and states. Additionally, vendors providing education technologies use different formats, data schema, and elements to organise and store student data as no single standard can be enforced upon them to comply with. Thus, disparate data scattered across different systems with varying degrees of accessibility and usability poses challenges to educators and other stakeholders in the education sector in assessing the impact of education technologies on the learning process. Moreover, lack of data standard compliance further limits the potential for imposing privacy law as is the case with the General Data Protection Regulation (GDPR), which provides sound privacy practices that are, on the one hand, enabled by data interoperability, and on the other, imposing vendors to maintain data transparency.
  
  \item Lack of transparency - The scale, complexity, and the number of providers pose challenges for schools to have a comprehensive list of the data that is generated about and by students. Lack of data interoperability standard, which would simply the process of compiling, organising, and documenting what information is generated about students, how this information is used and by whom, makes data auditability difficult, near impossible, to achieve, less so to be useful to the instruction and learning processes. 
  
  \item Security - Due to lack of transparency with regards to how various education technology vendors organise and structure their data, it becomes near impossible to understand the security that they ensure for the data their products and services generate. A recent case  presents the fears that exist with regards to school data (Cameron, 2017). Schoolzilla, a K-12 education data service provider, detected a data breach, which exposed the personal data of 1.3 million K-12 students. Some argue (Hand, 2018) that the current available cybersecurity protocols are inefficient to secure student data while at the same time allow its use for analytics without compromising any part of it. Having inadequate access to all available data further limits the potentials to benefit fully from it.
  
\end{itemize}

We can thus see that schools are finding it increasingly difficult to obtain data about their own students and are becoming isolated due to a lack of knowledge on how vendors are processing this data.

\section{The need for a solution}

Interoperability challenges between vendors and schools at local, district, state, or federal level further pose barriers not only to cohesive data management and sharing but also lacks technological privacy infrastructure and accountability [Zeide, 2014; Common Sense Media Research, 2018]. While digitizing and collecting student data is not new [Fitzgerald, 2014], technological developments like cloud computing and the growing use of web applications, learning management systems that gather fine-grained student data, amplify the concerns about data transfer, storage, use, and analysis [Sultan, 2010].
\bigbreak
Concerns about the over-prevalence of fine-grained data collected about students [Zeide, 2017] and the risk of creating a 'permanent record' that can impede upon learners' futures [Cody, 2013] arise. The use of online platforms and applications in schools often provided by for-profit vendors, many of whom have unclear policies about data privacy and third-party sharing [Common Sense Education, 2018], generates continuous stream of data about students' behavior and performance at an unprecedented scale [Zeide, 2017]. Big data complicates traditional understanding of what constitutes sensitive information and what information serves an educational purpose. Data however, continues to drive decision-making of practitioners and vendors, marginalizing student dimensions of learning and equitable participation in curriculum design.
\bigbreak
The present challenges of fragmented data access and use, lack of student agency, accountability, and auditability to data access and use demand a socio-technological solution, one which can ensure that educational data can be used to help improve school processes while student privacy, agency and future opportunities are not diminished.
\section{Kratos}
Kratos aims to define a set of principles and guidelines that schools should follow and provides a platform for both schools and students to see what kind of data is being collected by vendors along with ways to ensure that there is an immutable log and auditability for student data changes, access, and use. The goal for students would be to see how their data is being used and above the age of 18, gain access to this data. We define this as Data Ownership and Visibility - how students can own the data after the age of 18 and how students can see what their data is being used for. The goal for the school would be to make disparate data from different vendors compatible with each other and we define this problem as Data Interoperability.

\textbf{INSERT KRATOS.JPG HERE}

\subsection{Data Interoperability}
A traditional approach to interoperability would be to force a given set of standards on third party vendors, so that they reference them. However, work done by standardization bodies like SIF and CEDS has shown that this is at best partially effective since there is no financial or social incentive for vendors to migrate to the proposed set of standards. Kratos attempts to solve this issue by proposing a solution where different fields described by different vendors can be mapped to a single underlying scheme defined by Kratos and third party vendors would not have to worry about using disparate fields.

\textbf{INSERT DATA ARCHITECTURE.JPG HERE}

\bigbreak
As an example, let us assume the underlying fields defined by Kratos are Grade\_A, Grade\_B and Grade\_C. If a specific vendor has a different name for the same field (example GradeA, GradeB and GradeC), Kratos would ingest the data from the vendor defined fields and convert it to the ones defined by Kratos. This should be done across different formats since the reports database administrators receive right now are formatted as JSON, CSV or are in the form of Excel sheets.

\bigbreak
Kratos proposes that this be done in the form of a templated script that can be written for every vendor. Building a common script for all the vendors poses a challenge since the number of vendors and the different fields and standards they follow are ever changing. This templated script would be triggered automatically by Kratos each time it receives an incoming report and the report itself will be parsed to understand which vendor it originates from.

\bigbreak
The fields that Kratos defines will be adapted from the existing SIF standard followed by schools in Cambridge and would also have routine input from various experts on the subject. The number of fields however would change since SIF has over 700 different fields and it is not possible to accurately map all these fields. Kratos defines the idea of a "bucket" - a collection of data fields collated together as a single entity to make it easier for students and administrators to monitor them. Common examples of buckets would be PII (Personally Identifiable Information), Grade Reports (containing grade reports for each subject) and Attendance Records (recording the class wise attendance of the student). To formalise the type of data buckets it is imperative to develop a data taxonomy that comprises the existing school data standards and data elements according to which different education technology providers collect and organise their data. 

\bigbreak
The models and scripts developed as part of enabling interoperability will be open source and subject to continuous updates. We would also be having a complete code audit before releasing the system in a production environment to ensure the model performs and behaves the way it was designed and intended to.

\bigbreak
The buckets defined by Kratos can be used in multiple ways: students can have a better understanding of how their data is being used, parents can know what kind of data is being collected about their children, and administrators can ensure that sensitive information is not being shared with third-party vendors. After legal age of consent, this control would be given to the hands of students and they can decide if they wish to continue sharing data with these parties.

\subsection{Data control}
In conventional systems, data control or 'ownership' can be proved with the help of an access token but this provides no guarantee on when the owner came into possession of the data. In order to attest 'ownership' at a specific point in time, we need time-stamping services like those described in \hyperref[sec:1]{[1]}, \hyperref[sec:2]{[2]} and \hyperref[sec:3]{[3]}. A time-stamping service requires something to be committed along with the time-stamp and Kratos envisions this to be a cryptographic hash which also acts as an access token which vendors can use to access any student-generated data.
\bigbreak
The aim of using a cryptographic hash like SHA-3 \hyperref[sec:6]{[6]} is to ensure a uniquely random reference to the data to avoid out-of-channel data leaks. Kratos suggests using an element of randomness like a salt to generate the hash in order to have the ability to revoke tokens by regenerating randomness. Kratos enforces that all schools encrypt their data before creating access tokens to mitigate the risk of loss/theft of data. Past studies like \hyperref[sec:4]{[4]} and \hyperref[sec:5]{[5]} show that firms are willing to circumnavigate laws to collect data and encrypting data by design ensures that no third party can have access without being granted so explicitly.
\bigbreak
Kratos does not specify at which level data needs to be encrypted - it can be at the school level or at the student level depending on existing frameworks and rules surrounding the school. If the school chooses to encrypt student data on behalf of the student, Kratos enforces that the school use a unique key for each individual to minimize the risk of key theft. In addition to this, Kratos suggests that schools store their encrypted data in a distributed file storage system like IPFS to ensure data redundancy and availability in case of a setback. Storing data on IPFS also makes it easier to create timestamps since it is sufficient to reference the IPFS pointer instead of potentially hashing the whole data.
\bigbreak
In the event a user wants to revoke access to a particular vendor, he could do so by changing the encryption key or by changing the randomness used to generate the access token. We suggest users do not regenerate encryption keys but Kratos does not enforce this and will provide end users with the option to choose between the two.
\bigbreak
Since schools have their own sets of policies, Kratos does not strictly enforce a set of practices for users to follow. This ensures that adoption of Kratos is not constrained by a certain set of rules. At the same time, Kratos defines a set of minimum requirements to be on board to ensure good practices on data protection are followed.
\bigbreak
Kratos also does not specify how and where the commitments need to be stored and leaves it to schools to provide their feedback. Potential solutions include a centralized time-stamping server, a permissioned blockchain with the different schools as the nodes and simple time-stamping commitments to an existing blockchain. All three options have their benefits and constraints and Kratos would arrive at a final recommendation depending on what schools support best.
\bigbreak

\subsection{Student agency and participation}
The UNESCO framework for educational planning states that "the concern of planners is twofold: to reach a better understanding of the validity of education in its own empirically observed specific dimensions and to help in defining appropriate strategies for change" (Haddad, 1995, pp. 5-6).
\bigbreak
While summative and cumulative assessments provide "empirically observed specific dimensions" about student academic performance, student agency and active participation is equally required in order for policy and education to design "strategies for change". Growing use of education technologies enables data-driven decision-making [Gibson et al., 2015]. Technology-mediated instruction and assessment tools with learning analytics functionality track and diagnose student progress. Most educational technologies can interpret and equip educators with information via digital dashboards and 'skill meters'[Baker, 2016; New, 2016]. Fine-grained and continuously accumulated data about student behavior and performance surpasses traditional notions of assessment [Zeide, 2017] posing limitations over student dimensions of learning and equitable participation in the curriculum design.
\bigbreak
Our prototype provides a graphical interface for student involvement in and accessibility to school data. While students and equally their parents can become acquainted with any changes and meanings of school data, students can also participate with personal feedback to the learning process. Student participation with personal perspectives and reflection are integral to the learning process (Ackermann, 1996). Some reflection is invariably carried out through questionnaire surveys examining school climate (Holahan and Batey, 2019). Both school climate surveys and the proposed student feedback application provide flexible selection of questionnaires and measurements with a common goal to improve school climate. However, the proposed application enables not only feedback from older students (Ibid.) but from all students, provided that the feedback addresses the goal to encourage perspective-taking and reflection that are deemed necessary to the learning process (Kegan, 1982). The application further enables student agency and control over the frequency, depth, and nature of the feedback provided that it directly reflects the learning experience.

\subsection{Data literacy}
Data literacy is the ability to understand, create, and communicate data as information. Beyond that, in increasingly data-driven systems it becomes imperative not only to understand data but to interact with it and participate in the decision-making processes that it increasingly begins to impact. While data provides insight about how systems operate, the inferences drawn from it is without context. Fine-grained data collected over long periods of time can lead to negative impact on individual well-being and pose limitations over future opportunities [Gasser et al., 2018]. The growing digitization of schools create learning environments that enable constant data generation, the access to and use of which becomes hard to control, audit, and account for. The risk of digitized school environments constantly generating data and algorithmic assessments becoming permanent record that at any one point in time may limit individual opportunities leads us to prioritise on developing a socio-technological solution that enables applied data literacy for students at a young age. \textit{a note on Parents for Privacy Consortium and Data Quality Standards who make such efforts to educate students and parents about school data and that we (Kratos) complement these efforts.}

\section{Discussion and future work}
In this paper we propose Kratos, a decentralised data management system, which enables data interoperability, student agency, applied data literacy, and student participation in the curriculum design. This proposal envisions an ideal techno-social solution that empowers the individual student by providing more control over their school data through data transparency, accountability, and auditability. At the same time, continuous legal (GDPR, 2016; Colorado Student Data Transparency and Security Act, 2016; Kelly, 2019) and policy (Student Data Privacy Consortium, 2018) efforts are being made in the right direction that will further help establish Kratos as a tool integral to the school ecosystem. 

To leverage decentralisation of a large scale data management system, Kratos shows that as learning analytics are crucial for educators to improve on instruction and pedagogy, the platform prioritises student agency, privacy, and control over their data.  

Unlike other sectors such as retail (Lu and Xu, 2017), finance (Ito et al., 2017), and medicine (Acbo et al., 2019), education is still not fully leveraging from the digital transformation it is undergoing. In medicine, for instance, while data interoperability is important in order for various stakeholders to access comprehensive patient records, decentralised technologies have been used to enable such access while prioritising patient agency, privacy, and security (Agbo et al., 2019). 

On the other hand, while data analytics provide insight about school processes, not everything that is measured is meaningful just as not everything meaningful can be given a numeric value. Students’ perspectives and reflection should equally count next to data-driven decision-making. Moreover, the sophistication of data analytics and long-term data collections further increase the risks of individual privacy. The risk of being publicly exposed can lead to the risk of being embarrassed or discriminated (Altman et al. 2018). Being aware of such risks can diminish one’s sense of freedom for self-expression and in the case with children and young people - the freedom to simply try things out and make mistakes. Therefore, a decentralised data management system that prioritises student agency, privacy, and control over data, becomes crucial to the development of a safe space in which learners can learn and voice their perspectives without fear. 

Future work includes three distinctive steps. First, we plan to develop a comprehensive taxonomy of school data. This will serve to develop a comprehensive data vocabulary for applied data literacy and identify data elements. The second step involves developing prototype and formalising platform sections and functionality. And step three involves carrying out user studies to examine student experience, develop and fine-tune UI. 


\section{Acknowledgement}
We thank Cambridge Public School district, Student Data Privacy Consortium, Acess for Learning (SIF) for their support. We thank Rayner Ng Jing Kai from Yale OpenLab (NUS) for the mockup designs, Mario Galea from  Geoffrey?

\section{References}
\begin{enumerate}
    \label{sec:1}
    \item Todd, P. OpenTimestamps: Scalable, Trustless, Distributed Timestamping with Bitcoin (2016). URL \url{https://petertodd.org/2016/opentimestamps-announcement}
    \label{sec:2}
    \item Szalachowski, P. (2018). Towards more reliable Bitcoin timestamps. arXiv preprint arXiv:1803.09028.
    \label{sec:3}
    \item Massias, H., Avila, X. S., \& Quisquater, J. J. (1999). Design of a secure timestamping service with minimal trust requirement. In the 20th Symposium on Information Theory in the Benelux.
    \label{sec:4}
    \item Zetter, K. Google Collected Data on Schoolchildren without permission (2016). URL \url{https://www.wired.com/2015/12/google-collected-data-on-schoolchildren-without-permission/}
    \label{sec:5}
    \item Dissent, Back-To-School Revolt in Springfield? Employees balk over using Google Drive as evidence of massive privacy breach mounts (2018).  URL \url{https://www.pogowasright.org/back-to-school-revolt-in-springfield-employees-balk-over-using-google-drive-as-evidence-of-massive-privacy-breach-mounts/}
    \label{sec:6}
    \item Bertoni, G., Daemen, J., Peeters, M., Van Assche, G., \& Van Keer, R. (2012). Keccak implementation overview. URL  \url{http://keccak.neokeon.org/Keccak-implementation-3.2.pdf}.
    \label{sec:7}
    \item CPSD. District Agreements Listing (2019). URL  \url{https://sdpc.a4l.org/district_listing.php?districtID=457}
    \label{sec:8}
    \item Colins, A., and Halverson, R. (2018).\textit{ Re-thinking education in the age of technology: The digital revolution and schooling in America}. Teach Zeide, E. (2016).
    \label{sec:9}
    \item 19 Times Data Analysis Empowered Students and Schools: Which Students Succeed and Why?.
    \label{sec:10}
    \item Colins, A., and Halverson, R. (2018).\textit{ Re-thinking education in the age of technology: The digital revolution and schooling in America}. Teachers College Press: New York.
    \label{sec:11}
    \item Cody, A. (2013). Will the data warehouse become every student and teacher's 'permanent record'? \textit{Education Week}, May, 20.
    \label{sec:12}
    \item Gibson, D. C., Webb, M., and Ifenthaler, D. (2015). Challenges of big data in educational assessment.\textit{Proceedings of the IADIS International Conference of Exploratory Learning in Digital Age}, 92-100.
    \label{sec:13}
    \item Sultan, N. (2010). Cloud computing for education: a new dawn? 30 International Journal of Information Management  109.
    \label{sec:14}
    \item Fitzgerald, B. (2014). Data collection isn't new. And it predates common core.Funny Monkey.
    \label{sec:15}
    \item Zeide, E. (2016). 19 Times Data Analysis Empowered Students and Schools: Which Students Succeed and Why?
    \item Kegan, R. (1982). \textit{The evolving self: Problem and process in human development}. Cambridge: Harvard University Press.
    \item Ackermann, E. (1996). Perspective-Taking and object Construction. \textit{In Constuctionism in Practice: Designing, Thinking, and Learning in a Digital World} (Kafai, Y., and Resnick, M., Eds.). Mahwah, New Jersey: Lawrence Erlbaum Associates. Part 1, Chap. 2. pp. 25-37.
    \item Gibson D.C., Webb M.E., Ifenthaler D. (2019) Measurement Challenges of Interactive Educational Assessment. In: Sampson D., Spector J., Ifenthaler D., Isaías P., Sergis S. (eds) Learning Technologies for Transforming Large-Scale Teaching, Learning, and Assessment. Springer, Cham.
    \item Agbo, C. , Mahmoud, Q., and Eklund, J. (2019). Blockchain technology in healthcare: A systematic review. \textit{Healthcare, 7\textit{ (56)}}, 1-30.
    \item Lu, Q. and Xu, X. (2017). Adaptable blockchain-based systems: A case study for product traceability. \textit{IEEE Software, 34\textit{ (6)}}, 21-27.
    \item Ito, J., Narula, N., and Ali, R. (2017). The blockchain will do to the financial system what the Internet did to the media. \textit{Harvard Business Review}, March 23. Available from: https://hbr.org/2017/03/the-blockchain-will-do-to-banks-and-law-firms-what-the-internet-did-to-media
    \item U.S. Department of Education, Office of Educational Technology (2012). \textit{Enhancing teaching and learning through educational data mining and learning analytics: An issue brief}. U.S. Department of education: Washington, D.C. 
    \item EU General Data Protection Regulation (GDPR): Regulation (EU) 2016/679 of the European Parliament and of the Council of 27 April 2016 on the protection of natural persons with regard to the processing of personal data and on the free movement of such data, and repealing Directive 95/46/EC (General Data Protection Regulation), OJ 2016 L 119/1.
    \item Kelly, M. (2019). New privacy bill would give parents an ‘Eraser Button’ and ban ads targeting children. \textit{The Verge}, March 12.

\end{enumerate}
\begin{figure}
    \centering
    \includegraphics[width=15cm]{{"images/Kratos.jpg}}
    \caption{Kratos Architecture}
    \label{fig:my_label}
\end{figure}
\end{document}
    
